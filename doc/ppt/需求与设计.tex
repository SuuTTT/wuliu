\documentclass{beamer}

\begin{document}

\begin{frame}
\frametitle{需求文档}
\tableofcontents
\end{frame}

\section{业务描述和需求}
\subsection{需求概述}
\begin{frame}
    \frametitle{需求概述}
    本需求的核心目标是为物流系统实现一个最优调配策略算法。给定输入的订单列表,在知道库存信息和运输成本的情况下,算法能输出最优的调配策略列表。优化目标是在满足既定约束的条件下,最大限度地降低总成本,同时提高订单的满足率。
\end{frame}

\subsection{需求的输入数据}
\begin{frame}
    \frametitle{需求的输入数据}
    - 订单列表:通过调用接口(getZytpcl)输入给算法。订单信息包括订单号、商品内码、需求数量、最晚商品调配完成时间、订单的仓库列表(包括仓库编号、仓库库存、运输时间)。
    - 库存和运输成本:这些信息是通过调用接口/queryYscb /ckylcxByUTC得到的。
\end{frame}

\subsection{需求的输出数据}
\begin{frame}
    \frametitle{需求的输出数据}
    - 最优调配策略。具体的输入输出格式,详见接口描述部分。
\end{frame}

\subsection{时间约束和其他约束}
\begin{frame}
    \frametitle{时间约束和其他约束}
    - 满足所有订单:订单收到的商品总量大于等于其请求的


\end{document}
